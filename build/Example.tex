%---------------------------------------------------------------------
% TJAA - Example Article - Example.tex
%
% This file format has been adapted from Springer's llncs proc package.
%
% Producer: Springer
% Adapted by Sinan Kaan Yerli, ODTU Fizik Bolumu, Ekim 2010.
%
% Version: 2015-06-15 - 1.1
% Version: 2019-08-01 - 2.0
% Version: 2021-04-01 - 2.1
% Version: 2022-01-01 - 2.2
% 	- added Example-Turkce file using \TRlang
% 	  which explains multi language abstract
% 	- updated explanations in manuscript meta-data notes
% 	- added sections: Astro. Macros, Astro. Journals
% 	- added description for new columntypes
% 	- added section for auth-year commands
% Version: 2022-01-22 - 2.3
%	- added Disclosure section
% Version: 2024-09-01 - 2.4
%	- updated author & address usage with new commands
%	- updated abstract description
%	- updated keyword description
%	- added descr. for new astronomical macros
%---------------------------------------------------------------------
\documentclass[usenatbib]{tjaa}

%---------------------------------------------------------------------
% Türkçe karakterler ile yazmak ve Türkçe heceleme icin aşağıdaki
% paketin etkin olması zorunludur.
\usepackage[utf8]{inputenc}
%%%%% AUTHORS - PLACE YOUR OWN PACKAGES HERE %%%%%
\usepackage{lipsum}
\newsavebox\verbbox

%%%%%%%%%%%%%%%%%%%%%%%%%%%%%%%%%%%%%%%%%%%%%%%%%%%%%%%%%%%%%%%%%%%%%%
%%%%                                                              %%%%
%%%% PLEASE DONT DELETE LINES CONTAINING "%%%%TJAA-OZEL"          %%%%
%%%%                                                              %%%%
%%%% PLEASE DONT CLEAR -OR- MOVE OUT OF THE LINE THE TEXTS        %%%%
%%%%   CONTAINING "%%%%TJAA-OZEL"                                 %%%%
%%%%                                                              %%%%
%%%% IF YOU HAVE YOUR OWN TEX FILE, IN THE SAME WAY USED IN       %%%%
%%%% THIS FILE POPULATE CORRESPONDING LINES OR LINE ENDINGS       %%%%
%%%% WITH "%%%%TJAA-OZEL" **MANUALLY**.                           %%%%
%%%%                                                              %%%%
%%%% ------------------------------------------------------------ %%%%
%%%% LUTFEN "%%%%TJAA-OZEL" SATIRLARINI SILMEYIN                  %%%%
%%%%                                                              %%%%
%%%% LUTFEN SATIR SONLARINDAKI "%%%%TJAA-OZEL" BOLUMLERINI        %%%%
%%%%   SILMEYIN YA DA SATIRDAN KOPARMAYIN                         %%%%
%%%%                                                              %%%%
%%%% ONCEDEN KODLANMIS BIR TEX DOSYANIZ VARSA BU DOSYADA          %%%%
%%%% KULLANILDIGI GIBI "%%%%TJAA-OZEL" SATIRLARINI VE SATIR SONU  %%%%
%%%% EKLENTILERINI **ELLE** EKLEYIN                               %%%%
%%%%                                                              %%%%
%%%%%%%%%%%%%%%%%%%%%%%%%%%%%%%%%%%%%%%%%%%%%%%%%%%%%%%%%%%%%%%%%%%%%%

%%%%TJAA-OZEL:BASLIK%
% NOTES (title):
% 1) If you need to put a \newline in the title
%    then you HAVE TO specify the short title
%    DONT USE [] (short title) if your title not overlapping short author
\title[Short Title max 45 chars]{Método para la selección de modelos LSTM o GRU \newline en series de tiempo basada en el análisis comparativo\newline de su sostenibilidad y desempeño}%%%%TJAA-OZEL:TITLE%

% NOTES (authors):
% 1) For single author don't number single affiliation addreess.
% 2) If you need two or more lines of authors, use \newauthor (see below usage)
% 3) ORCID is required for ALL authors
% 4) Short author should be either A.U. Thor et.al -or- A. U. Thor
% 5) Depending on the language \others will be either "et al." or "v.ark."
\author[F. Author \others]{%
Juan Felipe Campos Adanaqué\autid{1}{0009-0003-6571-3554},
Enrique Joau Matencio Flores\autid{2}{0009-0008-8776-8501},
\newauthor
Anthony Alvaro Janampa Calderon \autid{3}{0009-0005-9097-8858}
\\
% NOTES (List of institutions):
% 1) Don't put \\ at the last institution
\adrid{1}Universidad Peruana Union, Peru,Lima%
}%%%%TJAA-OZEL:AUTHOR%
% These dates and numbers will be filled out by the publisher
\date{Accepted: XXX. Revised: YYY. Received: ZZZ.}
%
\renewcommand{\pubyear}{0000}
\renewcommand{\volume}{0}
\renewcommand{\issue}{0}

% NOTES (language):
% 1) Default language of TJAA is ENGLISH.
% 2a) If your manuscript is in ENGLISH then leave both commands as commented
%\ENlang
% 2b) If your manuscript is in TURKISH then uncomment only below command
%\TRlang

\begin{document}
% Don't change these 3 lines
\label{firstpage}
\pagerange{\pageref{firstpage}--\pageref{lastpage}}
\maketitle{M00-0000}

\begin{abstract}%%%%TJAA-OZEL:ABS%
% NOTES (abstract):
% >>>>>IF LANGUAGE IS IN ENGLISH<<<<< | >>>>>IF LANGUAGE IS IN TURKISH<<<<< | 
% \begin{abstract}%%%%TJAA-OZEL:ABS%  | \begin{abstract}%%%%TJAA-OZEL:ABS%  
% ENGLISH TEXT, ENGLISH TEXT,         | TURKISH TEXT, TURKISH TEXT,
% ENGLISH TEXT, ENGLISH TEXT,         | TURKISH TEXT, TURKISH TEXT,
% ENGLISH TEXT, ENGLISH TEXT,         | TURKISH TEXT, TURKISH TEXT,
% \end{abstract}                      | \ozet
%                                     | ENGLISH TEXT, ENGLISH TEXT,         
%                                     | ENGLISH TEXT, ENGLISH TEXT,          
%                                     | ENGLISH TEXT, ENGLISH TEXT,         
%                                     | \end{abstract}
%
% 0) Please follow above example for abstract
% 1) If your language is in ENGLISH
%    (a) put your abstract below and
%    (b) delete (2) option text below
Este proyecto se centra en la selección de modelos de redes neuronales recurrentes (RNN) LSTM y GRU para predicciones de series temporales, centrándose en su sostenibilidad y rendimiento. Aunque ambos modelos son eficaces para manejar dependencias a largo plazo en datos secuenciales, presentan diferencias en el consumo de recursos y la precisión. LSTM, 
Con su estructura más compleja, suele ser más preciso, pero requiere una mayor capacidad computacional, lo que limita su uso en entornos con recursos limitados. Por otro lado, GRU, al tener menos parámetros, es más eficiente y adecuado para aplicaciones donde la velocidad y el uso de recursos son cruciales. 
El estudio aplica una metodología preexperimental, evaluando ambos modelos en un contexto controlado con datos extraídos de PhishTank. Se analizarán métricas de rendimiento, como el error cuadrático medio (RMSE) y el error absoluto medio (MAE), para medir la precisión y eficiencia de cada modelo en la predicción del phishing. 
% 2) If your manuscript is in TURKISH (otherwise ignore below usage)
%    a) Write TURKISH abstract above
%    b) "uncomment" the command '\ozet' below
%    c) Put ENGLISH abstract below the command
%\ozet
El punto de referencia incluirá pruebas con datos ruidosos y se centrará en determinar qué modelo es más eficiente en el uso de recursos a largo plazo. Además, se utilizarán herramientas como Power BI para visualizar resultados y analizar la sostenibilidad de los modelos en diferentes escenarios.
\end{abstract}
% NOTES (Keywords):
% 1) Select minimum one and maximum six entries from the approred list 
% 2) Don't make up new ones.
% 3) THEY HAVE TO BE ALL IN ENGLISH
%
\begin{keywords}
% W A R N I N G : ***** N O  T U R K I S H  K E Y W O R D S *****
Series de Tiempo -- Sostenibilidad -- Rendimiento
% W A R N I N G : ***** N O  T U R K I S H  K E Y W O R D S *****
\end{keywords}

%%%%TJAA-OZEL:BILDIRI%
% - - - - - - - - - - - - - - - - - - - - - - - - - - - - - - - - - - -
\section{Introduction}
En la era de la inteligencia artificial y el aprendizaje automático, las redes neuronales recurrentes (RNN), en particular las arquitecturas Long Short-Term Memory (LSTM) y Gated Recurrent Units (GRU), han emergido como herramientas clave para modelar series temporales complejas, las cuales juegan un papel crucial en diversas industrias, como las finanzas, la salud y la logística. Estos modelos son capaces de capturar dependencias temporales en secuencias de datos, permitiendo predicciones precisas que pueden influir significativamente en la toma de decisiones estratégicas. Sin embargo, a medida que el volumen de datos aumenta y las necesidades de procesamiento en tiempo real se vuelven más exigentes, surge una preocupación sobre la sostenibilidad computacional de estos modelos, ya que no solo es importante alcanzar alta precisión, sino también hacerlo de manera eficiente en términos de recursos computacionales [2]. Aunque tanto LSTM como GRU han demostrado ser efectivos en el pronóstico de series temporales, hay importantes desafíos que persisten en relación con la eficiencia y sostenibilidad de estos modelos a largo plazo. Los modelos LSTM son conocidos por su capacidad de manejar dependencias a largo plazo, pero debido a su arquitectura más compleja (que incluye múltiples puertas como la de entrada, salida y olvido), consumen más recursos computacionales, lo que limita su aplicabilidad en entornos con restricciones de hardware o que requieren procesamiento en tiempo real [3]. Por otro lado, los modelos GRU, una variante simplificada, utilizan menos recursos y pueden ofrecer un rendimiento comparable, aunque en ciertos escenarios no logran la misma precisión que los LSTM [4]. Sin embargo, a pesar de estos avances, existe una falta de estudios comparativos exhaustivos que evalúen ambos modelos no solo en términos de precisión, sino también en cuanto a su coste computacional y sostenibilidad operativa en el tiempo, especialmente en aplicaciones de gran escala o continuo uso de datos [5]. El objetivo principal de esta investigación es realizar una comparación sistemática entre los modelos LSTM y GRU, abordando tanto su rendimiento en la predicción de series temporales como su eficiencia en el uso de recursos computacionales. La investigación busca proporcionar una metodología clara para ayudar a los desarrolladores y científicos de datos a seleccionar el modelo más adecuado según las características específicas de su proyecto y las necesidades de optimización de recursos, maximizando la eficacia y minimizando el consumo energético a lo largo del tiempo [6]. La capacidad de tomar decisiones basadas en predicciones precisas y eficientes es fundamental en sectores como las finanzas, donde los pronósticos pueden influir en la rentabilidad, o en la salud, donde los errores pueden tener consecuencias críticas. Este estudio no solo busca mejorar la precisión de las predicciones, sino también optimizar el uso de recursos computacionales, contribuyendo a la implementación sostenible de tecnologías de inteligencia artificial en diversos sectores, brindando soluciones más escalables y eficientes.

\section{Related Works}
El uso de redes neuronales recurrentes ha sido ampliamente estudiado para abordar problemas de predicción en series de tiempo. LSTM fue presentado por Hochreiter y Schmidhuber en 1997 [5] como una solución al problema del desvanecimiento del gradiente, permitiendo que las redes neuronales mantuvieran información relevante por períodos más largos en datos secuenciales. Gracias a su capacidad para capturar dependencias temporales, LSTM ha sido utilizado con éxito en aplicaciones que van desde la predicción del precio de acciones hasta la detección de anomalías en grandes infraestructuras industriales [6]. Sin embargo, su estructura compleja, que incluye tres tipos de puertas —de entrada, de olvido y de salida—, implica un mayor consumo de recursos computacionales y tiempo de entrenamiento [7]. 

Por otro lado, GRU, introducido en 2014 por Cho et al. [8], es una variante más ligera y simplificada de LSTM, con solo dos puertas (de actualización y de reinicio), lo que reduce considerablemente el número de parámetros a entrenar. Diversos estudios han mostrado que GRU puede alcanzar una precisión comparable a LSTM, pero con tiempos de entrenamiento más cortos y un menor consumo de memoria [9]. Esto hace que GRU sea especialmente útil en aplicaciones de tiempo real, como la predicción de demanda energética, donde la velocidad y la eficiencia en la predicción son esenciales [10]. A pesar de esto, los estudios que comparan ambos modelos suelen enfocarse únicamente en la precisión, ignorando otros factores importantes como el uso de recursos y la sostenibilidad computacional [11]. Este estudio tiene como objetivo ofrecer una evaluación integral que incluya estos aspectos, proporcionando una base para seleccionar modelos no solo en función de la precisión, sino también de su eficiencia a largo plazo. 

\subsubsection{First Section}
\subsubsection{First Section}
\paragraph{First Section - First sub-sub-Section - Paragraph}
\lipsum[1]

\newpage
% - - - - - - - - - - - - - - - - - - - - - - - - - - - - - - - - - - -
\section{Methodology}

A new command is introduced to collect all astronomical objects into a
database:
\verb|\Aobj{Cyg X-1}|.
The following code
``\verb|Andromeda galaxy (\Aobj{M31}) ...|''
will produce
``Andromeda galaxy (\Aobj{M31}) ...''
and the object (M31) will be added to the object database.

Links to web pages can be given within the text using \verb|\url| command:
The command \verb|\url{https://tad.org.tr}| will produce
\url{https://tad.org.tr}.
For longer URLs you can merge the URL into text using the web format.
The command
\verb|\href{https://dergipark.org.tr/tjaa}{TJAA}|\\
will produce a link to \href{https://dergipark.org.tr/tjaa}{TJAA}.

Some of the common commands, properties and habits:
\begin{flushleft}
\begin{tabular}{@{}p{.15\columnwidth}p{.75\columnwidth}}
\verb|OK|   & a thin space, e.g.\ between numbers or between units
              and numbers; a line division will not be made
              following this space\\
\verb|--|   & en dash; two strokes, without a space at either end;
	      put in between numbers to increase readability\\
\verb*| -- |& en dash; two strokes, with  a space at either end\\
\verb|-|    & hyphen; one stroke, no space at either end\\
\verb|$-$|  & minus, in the text {\em only} \\
{\em Input} & \verb|21\,$^{\circ}$C etc.,|\\
            & \verb|Dr h.\,c.\,Rockefellar-Smith \dots|\\
            & \verb|1950--1985 \dots|\\
            & \verb|this -- printed on a computer|\\
{\em Output}& 21\,$^{\circ}$C etc., Dr h.\,c.\,Rockefellar-Smith \dots\\
            & 1950--1985 \dots\\
            & this -- printed on a computer
\end{tabular}
\end{flushleft}

% - - - - - - - - - - - - - - - - - - - - - - - - - - - - - - - - - - -
\section{Special text typefaces}

Normal type (roman text) need not be coded. Three different command
emphasizes the text:
\begin{flushleft}
\begin{tabular}{@{}p{.32\columnwidth}p{.6\columnwidth}}
\verb|{\itshape Text}|   & {\itshape Italicized Text}\\
\verb|{\em Text}|   & {\em Emphasized Text --
   if you would like to emphasize a {\em definition} within an
   italicized text (e.g.\ of a {\em theorem}) you should code the
   expression to be emphasized by} \verb|\em|.\\
\verb|{\bfseries Text}|& {\bfseries Important Text}\\
\verb|\vec{Symbol}| & Vectors may only appear in math mode. The default
   \LaTeX{} vector symbol has been adapted\footnotemark\
   to LLNCS conventions.\\[2pt]
 & \verb|$\vec{A \times B\cdot C}|\\
 & ... yields $\vec{A\times B\cdot C}$
\end{tabular}
\end{flushleft}
%\footnotetext{If you absolutely must revive the original \LaTeX{}
%design of the vector symbol (as an arrow accent), please specify the
%option \texttt{[orivec]} in the \texttt{documentclass} line.}

% - - - - - - - - - - - - - - - - - - - - - - - - - - - - - - - - - - -
\section{Astronomical Macros}

List of short cuts for astronomy. Note that all of the macros listed below are

\begin{figure}
\includegraphics[height=7cm]{img/Mapa mental proyecto creativo colorido .png}
\caption{Modelo LSTM}
\label{lstm}
\end{figure}


\begin{flushleft}
\begin{large}
\begin{tabular}{@{}p{.50\columnwidth}@{~}p{.40\columnwidth}@{}}
\verb|12 \fd     34\fh     56| & $12\fd34\fh56$ \\
\verb|12 \fm     34\fs     56| & $12\fm34\fs56$ \\
\verb|12 \fdg    34\fas    56| & $12\fdg34\fas56$ \\
\verb|12 \fp     3456        | & $12\fp3456$ \\
\verb|12 \farcm  34\farcs  56| & $12\farcm34\farcs56$ \\
\verb|12 \arcmin 34\arcsec 56| & 12\arcmin34\arcsec56 \\
\end{tabular}
\end{large}
\end{flushleft}
\newpage
\begin{flushleft}
\begin{large}
\begin{tabular}{@{}p{.50\columnwidth}@{~}p{.50\columnwidth}@{}}
\verb|1\degr                | & 1\degr \\
\verb|1\Fd  2\Fh  3\Fm  4\Fs| & $1\Fd 2\Fh 3\Fm 4\Fs$ \\
\verb|1\Fdg 2\Fam 3\Fas     | & $1\Fdg 2\Fam 3\Fas$ \\
&\\[-2pt]
\verb|1 {M,R,L,T}\solarxxx  | & $1\solarmass 1\solarrad 1\solarlum 1\solartemp$ \\
\verb|1 {unit}\perxxx       | & g\perone m\persq s\percube \\
\verb|\renkBV \renkEBV      | & $\renkBV=\renkEBV$ \\
\verb|\renkUB \renkEUB      | & $\renkUB=\renkEUB$ \\
\verb|12 \micron \ion{H}{II}| & $12\micron$ \ion{H}{II} \\
\end{tabular}
\end{large}
\end{flushleft}

% - - - - - - - - - - - - - - - - - - - - - - - - - - - - - - - - - - -
\section{Astronomical Journals}

List of journal abbreviations used in TJAA references. Note that to fit into
the page only a partial list shown here.
\begin{verbatim}
\aap \aapr \aaps \aat \actaa \afz \aj \ao \apj
\apjl \apjs \aplett \apss \araa \arep \aspc \azh
\baas \caa \cjaa \iaucirc \icarus \japa \jrasc
\mnras \na \nar \nat \pasa \pasj \pasp \planss
\procspie \qjras \rmxaa \sci \skytel \solphys
\sovast \ssr \tjaa \an \psj          \vark
\end{verbatim}
\begin{footnotesize}
\aap		~$\bullet$
\aapr		~$\bullet$
\aaps		~$\bullet$
\aat		~$\bullet$
\actaa		~$\bullet$
\afz		~$\bullet$
\aj		~$\bullet$
\ao 		~$\bullet$
\apj		~$\bullet$
\apjl		~$\bullet$
\apjs		~$\bullet$
\aplett		~$\bullet$
\apss		~$\bullet$
\araa		~$\bullet$
\arep		~$\bullet$
\aspc		~$\bullet$
\azh		~$\bullet$
\baas		~$\bullet$
\caa		~$\bullet$
\cjaa		~$\bullet$
\iaucirc	~$\bullet$
\icarus		~$\bullet$
\japa		~$\bullet$
\jrasc		~$\bullet$
\mnras		~$\bullet$
\na		~$\bullet$
\nar		~$\bullet$
\nat		~$\bullet$
\pasa		~$\bullet$
\pasj		~$\bullet$
\pasp		~$\bullet$
\planss		~$\bullet$
\procspie	~$\bullet$
\qjras		~$\bullet$
\rmxaa		~$\bullet$
\sci		~$\bullet$
\skytel		~$\bullet$
\solphys	~$\bullet$
\sovast		~$\bullet$
\ssr		~$\bullet$
\an		~$\bullet$
\psj		~$\bullet$
\tjaa\\\vark
\end{footnotesize}

% - - - - - - - - - - - - - - - - - - - - - - - - - - - - - - - - - - -
\section{Footnotes}

\paragraph*{\itshape Sample Input}~\\
Text with a footnote\verb|\footnote{The |{footnotes are numbered automatically.}\verb|}| and text continues \dots

\paragraph*{\itshape Sample Output}~\\
Text with a footnote\footnote{The footnotes are numbered automatically.}
and text continues \dots

% - - - - - - - - - - - - - - - - - - - - - - - - - - - - - - - - - - -
\section{Lists}

\paragraph*{\itshape Sample Input}
\begin{verbatim}
\begin{enumerate}
  \item First item
  \begin{enumerate}
    \item First nested item
  \end{enumerate}
  \item Second item
\end{enumerate}
\end{verbatim}
\paragraph*{\itshape Sample Output}
\begin{enumerate}
  \item First item
  \begin{enumerate}
    \item First nested item
  \end{enumerate}
  \item Second item
\end{enumerate}

% - - - - - - - - - - - - - - - - - - - - - - - - - - - - - - - - - - -
\newpage
\section{Floating environments: Figures \& Tables}
\begin{lrbox}{\verbbox}
\verb|\protect\citep{author2021}|
\end{lrbox}

They should be inserted after (not in) the  paragraph in which the figure is
first mentioned.
They will be numbered automatically.
Note that figures (and tables) become floating environments if they are not
attached to main text.
Therefore if you want to place the figure (or table) correctly
\textbf{do not add empty lines} before \verb|\begin{figure}| or
\verb|\begin{table}|.

If you have many floating blocks (figures or tables) you have two choices:
\begin{enumerate}
\item
Push all floats to the end \textbf{with no space between the blocks} in the
same order as they are referenced;
\item
Start arranging the location of floats from the very first float putting and
fixing its location couple of paragraphs before it is referenced first. This
way the floats will be -/+ 1 one page away from its expecting location.
\end{enumerate}
The image file formats are limited to only PNG or PDF types.

\subsection{Figures}

To leave the desired amount of space for the height of your figures, please
use the coding described below.
Please note that ``\verb|x|'' in the following coding stands for the actual
height of the figure:
\begin{verbatim}
\begin{figure}
\vspace{x cm}
%              (Use [ ] for short caption)
\caption[ ]{...text of caption...}
\end{figure}
\end{verbatim}
\paragraph*{\itshape Sample Input}
\begin{verbatim}
\begin{figure}
\vspace{2.5cm}
\caption{This is the caption of the figure
displaying nothing leaving a vertical space
of 2.5 cm.
If you need to use cited references in
caption use \protect command:
\protect\citep{author2021}.}
\end{figure}
\end{verbatim}
\paragraph*{\itshape Sample Output.}{\itshape Figure is placed at the top of the column} 
\begin{figure}
\vspace{2.5cm}
\caption{%
This is the caption of the figure displaying nothing leaving a vertical space
of 2.5 cm. If you need to use cited references in caption use the following
command: \usebox{\verbbox}.}
\end{figure}

% - - - - - - - - - - - - - - - - - - - - - - - - - - - - - - - - - - -
\newpage
\subsection{Tables}

Table captions should be treated in the same way as figure legends, except
that the table captions appear {\itshape above} the tables. 
\paragraph*{\itshape Sample Input}
\begin{verbatim}
\begin{table}
\caption{Critical $N$ values}
\begin{tabular}{rllllL{7mm}@{}C{7mm}@{}R{7mm}}
\hline     % No need to use \noalign{\smallskip}
${\mathrm M}_\odot$
  & $\beta_{0}$ & $T_{\mathrm c6}$ & $\gamma$
  & $N_{\mathrm{crit}}^{\mathrm L}$
  & L & C & R \\
\hline
% You dont need to use      \noalign{\smallskip}
% around the \hline command.
% It is provided internally.
 30 & 0.82 & 38.4 & 35.7 & 154 &1   &1    &1  \\
 60 & 0.67 & 42.1 & 34.7 & 138 &12  &12   &12 \\
120 & 0.52 & 45.1 & 34.0 & 124 &123 &123  &123\\
\hline
\end{tabular}
\end{table}
\end{verbatim}
\paragraph*{\itshape Sample Output:}
Table is placed at the top of this column. To place flowing text
correctly, start a new paragraph by pressing enter twice after the table.
\begin{table}
\caption{Critical $N$ values}
\centering
% Please dont use commands giving extra vertical or horizontal spacing.
% They are all deleted or converted to publisher standards.
\begin{tabular}{rllllL{7mm}@{}C{7mm}@{}R{7mm}}
% When \tabcolsep is used your table will take more space than expected.
%\setlength\tabcolsep{3pt}
\hline
${\mathrm M}_\odot$
  & $\beta_{0}$ & $T_{\mathrm c6}$ & $\gamma$
  & $N_{\mathrm{crit}}^{\mathrm L}$
  & L & C & R \\
\hline
 30 &0.82 &38.4 &35.7 &154 &1   &1   &1  \\
 60 &0.67 &42.1 &34.7 &138 &12  &12  &12 \\
120 &0.52 &45.1 &34.0 &124 &123 &123 &123\\
\hline
\end{tabular}
\end{table}

Two important notes:
(1) 4 new columntypes introduced (P, L, C, R) -- last three
shown here.
For example \verb|C{1cm}| centers the content of the column to 1 cm width.
(2) If you use \verb|dcolumn| option for \verb|tjaa| style, 3 new column types
(namely, \verb|d . ,|) can be used to align the numbers in the column content.

For further information you will find a complete description of the tabular
environment on p.~62~ff. and p.~204 of the {\em \LaTeX{} User's Guide \&
Reference Manual\/} by Leslie Lamport.

% - - - - - - - - - - - - - - - - - - - - - - - - - - - - - - - - - - -
\section{Symbols and Characters}

\subsection*{Special Symbols}

You may need to use special signs.  The available ones are listed in the {\em
\LaTeX{} User's Guide \& Reference Manual\/} by Leslie Lamport, pp.~41\,ff. We
have created further symbols for math mode (enclosed in \$):
\begin{center}
\begin{tabular}{l@{\hspace{1em}yields\hspace{1em}}
c@{\hspace{3em}}l@{\hspace{1em}yields\hspace{1em}}c}
\verb|\grole| & $\grole$ & \verb|\getsto| & $\getsto$\\
\verb|\lid|   & $\lid$   & \verb|\gid|    & $\gid$
\end{tabular}
\end{center}
\newpage

% - - - - - - - - - - - - - - - - - - - - - - - - - - - - - - - - - - -
\section{Acknowledging other parties}

Scientific works usually have to acknowledge many different institutions,
works, individuals, software etc.
In a proper acknowledgment, these should be listed according to the
importance and contribution, starting from highest to lowest, and group them:
projects, software, institutions, people.

Note also that acknowledgment section is not numbered and should be the last
section before references:
\paragraph*{\itshape Sample Input}
\begin{verbatim}
\section*{Acknowledgment}
\end{verbatim}

\section{Required Disclosures}
Authors are now required to disclose two extra information: (1) Conflict of
Interest, and (2) Contribution Percentage among the authors. Authors should
typeset the following section after the acknowledgement. Note that, the
following Turkish equivalents should be used for Turkish text, in order of
apparance: Açıklamalar, Çatışma Beyanı, Katkı Oranı.
\begin{verbatim}
\section*{Disclosures}
\paragraph{Conflict of Interest:} All authors
declare that they have no conflicts of interest.
\paragraph{Contribution:} All authors contributed
equally in writing the manuscript.
\end{verbatim}

If authors have conflict of interests then they could use the following
conflict of interest templates as examples:
\begin{verbatim}
- X has received a research grant from FIRM. 
- X is an employee of FIRM.
- X received part time salary for this work.
- X is on board of JOURNAL.
- X has a patent pending for PATENT.
- X declares no conflicts of interest.
\end{verbatim}

% - - - - - - - - - - - - - - - - - - - - - - - - - - - - - - - - - - -
\section{References}
\label{refer}

There are two ways to enter citations in \LaTeX{}:
(1) Citing with plain text inline in the manuscript using
\textit{pre-formatted list of references};
(2) Referring to a permanent database of references and calling with BibTeX
tools.
We recommend the later one to our colleagues, especially if you are a beginner
in writing scientific papers.
In this manuscript both of them will be explained.

There are three different citation presentation exists.
However, in this manuscript, only ``Author-Year'' style which is widely
used in astronomy, will be explained and therefore implemented in TJAA.
The other two are: number only and letter--number.

Entering citations and typesetting references are explained in
``{\em \LaTeX{} User's Guide \& Reference Manual\/}, Leslie Lamport, s.~71.''
In this manuscript only a simplified will be used.

\newpage
\subsection{Author-Year System}

Referenced cited within the text in parenthesis which contains author and
year.
Some of the examples of this usage are as follows:
(Smith 1970, 1980), (Ekeland et al. 1985, Theorem 2), (Jones and Jaffe
1986; Farrow 1988, Chap.\,2).
If the author is part of the text only the year could be put into the
parenthesis:
eg.\ Ekeland et.al.\ (1985, Section.\,2.1)
Reference list should include all the cited word and it has to be order with
the surname.
If there are more than one work for the same author then they should be listed
with the following rules:
\begin{enumerate}
\setlength{\hfuzz}{5pt}
\item
Single Author: List is sorted by date.
\item
Author and the same other authors: List is sorted by date.
\item
Author and different other authors: List has to be sorted alphabetically by
the other authors.
\end{enumerate}
If there are more than one cited work for the same author(s) then each work
has to be suffixed with sequential letter, ``a'', ``b'' etc.
eg.\ (Smith 1982a), (Smith 1982b), (Ekeland et al. 1982c).

\subsection{Author-Year commands}

For the following \verb|\bibitem[]{}| command created by BibTeX the following
list of citation usage is available through out the manuscript:\\
\verb|\bibitem[Auth1 et al. (2021)Auth1, Auth2,|\\
\verb|    and Auth3]{key21}...|
\small
\begin{flushleft}
\begin{tabular}{@{}p{.45\columnwidth}@{~}p{.50\columnwidth}@{}}
\verb|\citet{key21}|       & Auth1 et al. (2021) \\
\verb|\citep{key21}|       & (Auth1 et al., 2021) \\
\verb|\citet*{key21}|      & Auth1, Auth2, and Auth3 (2021) \\
\verb|\citep*{key21}|      & (Auth1, Auth2, and Auth3, 2021) \\
& \\
\verb|\citet[txt1]{key21}| & Auth1 et al. (2021, txt1) \\
\verb|\citep[txt1]{key21}| & (Auth1 et al., 2021, txt1) \\
\verb|\citep[txt1][]{key21}| & (txt1 Auth1 et al., 2021) \\
\verb|\citep[txt1][txt2]{key21}| & (txt1 Auth1 et al., 2021, txt2) \\
& \\
\verb|\citealt{key21}|     & Auth1 et al. 2021 \\
\verb|\citealp{key21}|     & Auth1 et al., 2021 \\
\verb|\citealt*{key21}|    & Auth1, Auth2, and Auth3 2021 \\
\verb|\citealp*{key21}|    & Auth1, Auth2, and Auth3, 2021 \\
& \\
\verb|\citetext{priv.\ comm.}| & (priv. comm.) \\
\end{tabular}
\end{flushleft}
\normalsize

\textbf{This system has been integrated into TJAA style and you don't need to
change anything in your manuscript; just continue using your habit of citing.}

\begin{center}
\Large\textbf{IMPORTANT NOTE:}\\
Below  you will find two different methods.\\
Choose one type of citing and referencing and\\
delete the other section completely.
\end{center}

\newpage
\subsection{Author-Year: Inline with text}

You have to assign an alias (embraced with curly brackets) to each reference
you put in `References' section with \verb|\bibitem| command.
You then use \verb|\cite{alias}| command to refer to the reference
inline within the text.

Note that in this `inline' method, author has full control of the format of
\verb|\bibitem|. However, since entries are written and maintained by hand,
copy-pasting from other source files might create inconsistencies which would
not easy to resolve for large number of entries.

\paragraph*{\itshape Example-1 -- Citing a reference: Input}
\begin{verbatim}
The results in this section are a
refined version of \citet{clar:eke1}; the
minimality result of Proposition~14 was the
first of its kind.
\end{verbatim}

\paragraph*{\itshape Example-1 -- Citing a reference: Output}
``\dots\ refined version of \citet{clar:eke1}; the minimality\dots''.

\paragraph*{\itshape Example-1 -- Building References: Input}
\begin{verbatim}
%      (don't forget {} string at the end)
\begin{thebibliography}{}
.
.
\bibitem[Clarke ve Ekeland, 1982]{clar:eke1}
Clarke, F., Ekeland, I.:
Nonlinear oscillations and boundary-value
problems for Hamiltonian systems.
Arch. Rat. Mech. Anal. {\bfseries 78} (1982)
315--333
.
.
\end{thebibliography}
\end{verbatim}

\paragraph*{\itshape Example-1 -- Building References: Output}
%%%%TJAA-OZEL:BIB%
\begin{thebibliography}{}
\bibitem[Clarke ve Ekeland, 1982]{clar:eke1} Clarke, F., Ekeland, I.: Nonlinear
oscillations and boundary-value problems for Hamiltonian systems.
Arch. Rat. Mech. Anal. {\bf 78} (1982) 315--333
\end{thebibliography}

\newpage
\subsection{Author-Year: with {\sc Bib}\TeX{}}

The only difference of this method from the \textit{inline} method is that
\verb|\bibitem| lines are generated automatically using an external command
(\verb|bibtext|).
Therefore, format of \verb|\bibitem| lines will also be standard too.
The standard for TJAA is adapted from Springer's {\sc Bib}\TeX{} style and it
is available through \verb|tjaa.bst| file and the command
\verb|\bibliographystyle{tjaa}| will handle the rest of the configuration.

Citation in this example will be exactly the same as in the previous one.

\paragraph*{\itshape Example-2 -- Citing a reference: Input}
\begin{verbatim}
The results in this section are a
refined version of \citet{clar:eke2}; the
minimality result of Proposition~14 was the
first of its kind.
\end{verbatim}

\paragraph*{\itshape Example-2 -- Citing a reference: Output}
``\dots\ refined version of \citet{clar:eke2}; the minimality\dots''.

\paragraph*{\itshape Example-2 -- Building References: Input}
\begin{verbatim}
\bibliographystyle{tjaa}
\bibliography{Example}
\end{verbatim}
\paragraph*{\itshape Example-2 -- Building References: Input} \verb|Example.bib| file.
\begin{verbatim}
@ARTICLE{clar:eke2,
 author  = {{Clarke}, F. and {Ekeland}, I.},
 title   = "{Nonlinear ... systems.}",
 journal = {Arch. Rat. Mech. Anal.},
 year    = 1982,
 volume  = 78,
 pages   = {315-333},
 tjaanote= {Prepared for TJAA example.
            Note: Even though each keyword=value
            ends with comma in previous lines,
            this line doesnt.}
}
\end{verbatim}
\paragraph*{\itshape Example-2 -- Building References: Output}
%%%%TJAA-OZEL:BIB%
\bibliographystyle{tjaa}
\bibliography{Example}

\subsection{Future of Bibliography Management}

Users might find it to difficult to maintain a bibliographic reference list in
a special formatted file with \verb|.bib| extension.
However, nowadays all bibliographical databases provide \verb|.bib| outputs.
Then, you just have to refer to these files using the following command:
\verb|\bibliography{tjaa2021}|.

Even though maintaining ``a reference database'' means to combine a few
different concepts together, once it is done for the first time, later usage
would be handled by combining different \verb|.bib| files in a
single command. For example, use the following command to combine your
previous TJAA submissions:
\verb|\bibliography{tjaa-M1,tjaa-2018,tjaa-2021}|.

In today's capabilities, please consider exporting your bibliographical
databases stored in \verb|.bib| files into a modern style software; one of
which is
\href{https://zotero.org/}{Zotero} and it is also integrated in
\href{https://overleaf.com/}{Overleaf}, a web based tool to collaboratively
write in \LaTeX{}, store, share and produce articles.
TJAA style is available in \href{https://overleaf.com/}{Overleaf}.

%%%%TJAA-OZEL:SON%
\label{lastpage}
\end{document}
